%%%%%%%%%%%%%%%%%%%%%%%%%%%%%%%%%%%%%%%%%
% Wenneker Assignment
% LaTeX Template
% Version 2.0 (12/1/2019)
%
% This template originates from:
% http://www.LaTeXTemplates.com
%
% Authors:
% Vel (vel@LaTeXTemplates.com)
% Frits Wenneker
%
% License:
% CC BYNCSA 3.0 (http://creativecommons.org/licenses/byncsa/3.0/)
% 
%%%%%%%%%%%%%%%%%%%%%%%%%%%%%%%%%%%%%%%%%

%
%	PACKAGES AND OTHER DOCUMENT CONFIGURATIONS
%

\documentclass[11pt]{scrartcl} % Font size

\input{structure.tex} % Include the file specifying the document structure and custom commands

%
%	TITLE SECTION
%

\title{	
	\normalfont\normalsize	
	\rule{\linewidth}{0.5pt}\\ % Thin top horizontal rule
	\vspace{20pt} % Whitespace
	{\huge Cuestionario de Teoría - 1}\\ % The assignment title
	\vspace{12pt} % Whitespace
	\rule{\linewidth}{2pt}\\ % Thick bottom horizontal rule
	\vspace{12pt} % Whitespace
}

\author{\LARGE Ignacio Vellido Expósito} % Your name

\date{\normalsize\today} % Today's date (\today) or a custom date

\begin{document}

\maketitle % Print the title

\subsection{Diga en una sola frase cuál cree que es el objetivo principal de la 
Visión por Computador.\newline Diga también cuál es la principal propiedad de las 
imágenes de cara a la creación algoritmos que la procesen.}

El objetivo principal es intentar ser capaz de procesar imágenes de forma igual
o mejor que el ser humano.\newline

Las imágenes terminan siendo cuantificaciones de la intensidad de luz, y esto se
puede asociar a una función. Esta función puede ser procesada por algoritmos
para diferentes usos.

\subsection{Expresar las diferencias y semejanzas entre las operaciones de 
correlación y convolución.\newline Dar una interpretación de cada una de ellas que en 
el contexto de uso en visión por computador.}

Como semejanza, ambas son técnicas de aplicado de máscaras a una imagen.
Además, con filtros lineales y gracias a la invariante de traslación, la salida 
no depende de la máscara, sino de la región en la que se aplica.\newline
	 
Respecto a sus diferencias, en correlación la máscara se aplica directamente, 
mientras que en convolución esta se invierte (por filas y columnas) previamente.

Esto hace que la correlación tenga propiedades adicionales de mayor utilidad de
cara a la visión por computador.
En caso de una máscara simétrica, ambas operaciones realizan el mismo cálculo.\newline

También se puede ver que la correlación se asemeja a la búsqueda de un determinado
modelo, mientras que la correlación corresponde a un filtrado.

\newpage

\subsection{¿Cuál es la diferencia “esencial” entre el filtro de convolución y 
el de mediana? Justificar la respuesta.}

Que el filtro de convolución aplica un cálculo sobre el vecindario de un píxel,
y sustituye su valor por el resultado.\newline

En cambio, la mediana no añade un nuevo valor como resultado de una operación, 
sino que selecciona un
píxel del vecindario (el situado en la mediana en una ordenación) y se lo asigna.

\subsection{Identifique el “mecanismo concreto” que usa un filtro de máscara 
para transformar una imagen.}

Un filtro de máscara realiza una operación sobre una zona local al píxel, no sobre
una parte global de la imagen.\newline

Esto se debe a que existe una relación entre regiones de píxeles que acaban
transmitiendo una cierta información en la imagen.

% (Localidad, usan información local y no global) La extracción de información a
% partir del vecindario de píxeles (no es correlación ni convolución, hay filtros
% de máscara no lineales)

\subsection{¿De qué depende que una máscara de convolución pueda ser 
implementada por convoluciones 1D? Justificar la respuesta.}

De que la máscara pueda descomponerse en vectores 1D, de forma que uno se
aplique por filas y otro por columnas. Esta propiedad es dependiente de
la máscara en concreto.\newline

Para saber si la máscara es separable se puede realizar su descomposición en 
valores singulares y comprobar si el primer valor singular es distinto de cero.

% \begin{equation}
% 	K = \sum_i \sigma_i u_iv_i^T
% \end{equation}

\subsection{Identificar las diferencias y consecuencias desde el punto de
vista teórico y de la implementación entre:\newline
a) Primero alisar la imagen y después calcular las derivadas sobre la imagen 
alisada\newline
b) Primero calcular las imágenes derivadas y después alisar dichas imágenes.\newline
Justificar los argumentos.}

Con a) conseguimos los detalles de una imagen suavizada, y con b) suavizamos
los detalles de una imagen normal.\newline

Esto se debe a que con a) eliminamos frecuencias altas de la imagen y luego 
mediante derivadas cogemos los detalles (bordes de objetos, también frecuencias altas) 
de la imagen resultado,
mientras que con b) el proceso es al contrario, obteniendo un resultado 
completamente distinto.

\subsection{Identifique las funciones de las que podemos extraer pesos
correctos para implementar de forma eficiente la primera derivada de una
imagen.\newline Suponer alisamiento Gaussiano.}

% Núcleos de la convolución separables por filas y columnas
% pag117

Podemos obtener los pesos tomando las derivadas de la función:
\begin{equation}
	G(x, y, \sigma) = \frac{1}{2\pi\sigma^2} e^{-\frac{x^2+y^2}{2\sigma^2}}
\end{equation}

\subsection{Identifique las funciones de las que podemos extraer pesos correctos
para implementar de forma eficiente la Laplaciana de una imagen.\newline Suponer
alisamiento Gaussiano.}

% Núcleos de la convolución separables por filas y columnas

Podemos obtener los pesos de una Laplaciana de Gaussiana a partir de la función:
\begin{equation}
	\nabla^2 G(x, y, \sigma) = 
	(\frac{x^2+y^2}{\sigma^4} - \frac{2}{\sigma^2}) * G(x, y, \sigma)
\end{equation}

\subsection{Suponga que le piden implementar de forma eficiente un algoritmo
para el cálculo de la derivada de primer orden sobre una imagen usando
alisamiento Gaussiano.\newline
Enumere y explique los pasos necesarios para llevarlo a cabo.}

Pregunta no contestada.


\subsection{Identifique semejanzas y diferencias entre la pirámide gaussiana y
el espacio de escalas de una imagen, ¿cuándo usar una u otra? Justificar los 
argumentos.}

% La Laplaciana se puede aproximar como diferencia de Gaussianas

La pirámide es un método que nos permite disminuir el tamaño de una imagen sin 
pérdida de información, mientras que el espacio de escalas aplica filtros en 
cada octava y genera las diferencias entre pares de ellos.\newline

El espacio de escalas es preferible por tener un menor coste computacional, pero
no es una aproximación perfecta y por tanto se pierde exactitud.

\subsection{¿Bajo qué condiciones podemos garantizar una perfecta reconstrucción
de una imagen a partir de su pirámide Laplaciana? Dar argumentos y discutir las
opciones que considere necesario.}

Siempre que en la costrucción de la pirámide se haya realizado un muestreo 
suficiente de la imagen (es decir, 
como mínimo al doble de la frecuencia de entrada), y que en la última capa se
hayan almacenado las frecuencias bajas residuales.

\subsection{¿Cuáles son las contribuciones más relevantes del algoritmo de Canny
al cálculo de los contornos sobre una imagen?\newline¿Existe alguna conexión entre las
máscaras de Sobel y el algoritmo de Canny?\newline Justificar la respuesta}

Canny aporta el uso de algoritmos de histéresis a la detección de bordes.\newline

Las máscaras de Sobel son máscaras de primera derivada con alisamiento que nos 
indican los cambios bruscos en los píxeles. Estos cambios corresponden a los 
detalles en la imagen, y entre ellos también se suelen encontrar los bordes.

\newpage

\subsection{Identificar pros y contras de kmedias como mecanismo para crear un
vocabulario visual a partir del cual poder caracterizar patrones.\newline
¿Qué ganamos y que perdemos?\newline Justificar los argumentos}

Kmedias es un mecanismo más simple y rápido que técnicas de machine learning,
aunque la elección del número de clústeres puede hacer variar los resultados obtenidos.\newline

Es un método que funciona bien cuando se tienen muchos datos y están mayoritariamente
agrupados.

% • Pros
% – Simple and fast
% – (Always) converges to a local minimum of the error function
% – Available implementations (e.g., in Matlab)
% •Cons
% –Need to pick K
% –Sensitive to initialization
% –Only finds “spherical”
% clusters
% –Sensitive to outliers

\subsection{Identifique pros y contras del modelo de “Bolsa de Palabras” como
mecanismo para caracterizar el contenido de una imagen. ¿Qué ganamos y que
perdemos?\newline Justificar los argumentos.}

Es un método que funciona bien para un conjunto suficientemente grande de imágenes,
pero la elección del tamaño del vocabulario acarrea un problema, ya que un número 
pequeño puede no ser suficiente para caracterizar el contenido, y uno muy grande
puede ser demasiado costoso computacionalmente.

\subsection{Suponga que dispone de un conjunto de imágenes de dos tipos de
clases bien diferenciadas. Suponga que conoce como implementar de forma
eficiente el cálculo de las derivadas hasta el orden N de la imagen. Describa
como crear un algoritmo que permita diferenciar, con garantías, imágenes de
ambas clases. Justificar cada uno de los pasos que proponga.}

Pregunta no contestada.

\end{document}
