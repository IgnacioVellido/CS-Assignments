\chapter{Descripción}

SynVic es un juego free-to-play de acción MOBA y perspectiva isométrica. En él los jugadores se enfrentan en partidas cortas mediante batallas 2vs2, donde los jugadores deben formar equipo entre sí y explotar al máximo las sinergias entre sus personajes. Tras elegir un personaje, los jugadores entran en una sala con un mapa y un objetivo determinado (elegidos aleatoriamente de entre una gran variedad), una vez ahí deberán votar para formar equipo con el jugador que más le interese para alcanzar la victoria. Adicionalmente, los jugadores cuentan con la posibilidad de adaptar algunos detalles de las habilidades al mapa seleccionado y a las características de cada equipo. De esta manera se fomenta la cooperación en la preparación de la partida y la selección de sinergias.

\vspace{\baselineskip}

Nuestro público objetivo son jugadores de entre 15 y 30 años. A diferencia de la mayoría de MOBAs, queremos atraer tanto a jugadores casuales como fans del género MOBA. Esperamos que los casuals se sientan atraídos por las partidas cortas, frenéticas, y con una curva de dificultad inicialmente baja, pero creciente. A los más gamers pretendemos convencerlos con un ecosistema competitivo rico y variado que permita enfrentarse con gente de igual habilidad. También incluiremos un sistema de logros de personajes para que puedan demostrar su maestría con ellos. Todo esto junto a un soporte constante conseguirá un buen apoyo de la comunidad

\vspace{\baselineskip}

Las microtransacciones van a estar limitadas a aspectos estéticos del juego, siguiendo la fórmula de éxito de otros MOBAs y evitando el modelo de negocio pay-to-win que tanto odio genera entre los jugadores. Al eliminar la barrera de pago inicial impulsará en gran medida la llegada de nuevos jugadores.

\vspace{\baselineskip}

Centraremos el estilo artístico es un 2D simplificado (posiblemente las físicas sean en 3D), es decir, se pretende no sobrecargar con detalles/efectos a los jugadores y que el juego resulte claro e intuitivo a primera vista. La idea es tematizar los elementos con colores característicos que permitan identificarlos con facilidad. Así conseguimos que el arte sea funcional, aportando una utilidad para el jugador y no ser meramente decorativo.