\chapter{Descripción del juego}

% ==============================================================================

\section{Características}
\subsection{General}
SynVic es un juego free-to-play de acción MOBA y perspectiva isométrica. En él los jugadores se enfrentan en partidas cortas mediante batallas 2vs2. Cada jugador elige el personaje con el que quiere jugar la siguiente partida y se le mete en un \emph{lobby} al azar con otros tres jugadores. De igual manera, se elige un mapa y objetivo al azar para esa partida, estando el objetivo asociado al mapa elegido. Una vez en el lobby, los jugadores toman parte en una votación para elegir a su compañero de equipo, teniendo en cuenta las habilidades de los distintos personajes y el objetivo de la partida.

\subsection{Multijugador}
El juego se centra en mapas 2vs2, con chat in-game y pre-game (en el lobby) para facilitar la cooperatividad. En el estado actual, no hemos implementado todavía el chat. En un futuro, se barajaría la adición de un sistema de chat de voz, si el chat de texto no fuera suficiente.

\vspace{\baselineskip}

También se incluye un sistema de clasificación de jugadores, separado en rangos, para facilitar un entorno competitivo para aquellos jugadores que lo deseen. De esta forma, serán emparejados con jugadores de su mismo nivel, haciendo que el \emph{match-making} sea justo. En la versión actual, no se ha implementado el sistema competitivo.

\subsection{Gameplay}
Queremos proporcionar una experiencia que sea a la vez dinámica, incluso caótica, y competitiva, centrada en la cooperación.

\vspace{\baselineskip}

La dinamicidad la proporcionamos mediante los elementos semi-aleatorios del juego, principalmente el \emph{match-making} aleatorio y la selección también aleatoria del mapa. Esto hará que cada partida sea diferente.

La cooperación la proporcionamos mediante las partidas 2v2. Los jugadores deben elegir bien a su compañero de equipo, estableciendo una estrategia para la victoria en función de la sinergias entre sus habilidades. Los diversos personajes del juego y sus habilidades estarán pensados para obligar a los jugadores a cooperar entre sí: un personaje tendrá un rol determinado y no será autosuficiente, de ahí la necesidad de cooperación.


% ==============================================================================

\section{Plataformas compatibles}
En principio nos centramos para las plataformas de PC Windows y Linux.

% ==============================================================================

\section{Público objetivo}
Nuestro público objetivo son jugadores jóvenes, de entre 12 y 35 años. Buscamos atraer, al mismo tiempo, jugadores acostumbrados a otros MOBA, que buscan una experiencia altamente competitiva, y a jugadores más \emph{casuals}, que buscan un juego con una curva de aprendizaje menos empinada que otros MOBA y donde poder disfrutar desde la primera partida.

% ==============================================================================

\section{Influencias}
Tomamos influencia de otros juegos del género MOBA (como \textit{Battlerite}, \textit{League of Legends} y \textit{DOTA2}) y derivados (como \textit{Overwatch}).

% ==============================================================================

\section{Cuestiones importantes}
% \subsection{¿De que va el juego?}
% \textit{En un párrafo, proporcionar una descripción general del juego.}

% \subsection{¿Cual es el objetivo principal del juego?}
% \textit{¿Que ha de conseguir el jugador dentro del juego para poder superarlo con éxito?.}

\subsection{¿Cuales son los aspectos diferenciadores del juego?}
% \textit{Con respecto a la competencia, ¿de que se diferencia nuestro juego?.}
En comparación con el resto de MOBAs, el juego se desmarca utilizando partidas cortas centradas en la cooperación. El hecho de que los jugadores seleccionen a su personaje sin conocer los personajes del resto de jugadores ni el objetivo de la partida hace que sea muy importante la capacidad de adaptación y cooperación con el compañero. Así, cada partida será diferente de la anterior y una nueva experiencia para el jugador.
	
% ==============================================================================
