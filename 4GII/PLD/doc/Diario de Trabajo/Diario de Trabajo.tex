\documentclass{article}

\title{Diario de Trabajo}
\date{}

\begin{document}
	\maketitle
	
\section*{Semana 1}
Elegimos el juego a desarrollar y establecimos sus características principales: MOBA, 2v2, hacer equipos con jugadores que no conoces y la gran importancia de las sinergias entre las habilidades de los jugadores.

\section*{Semana 2}
Terminamos de establecer las características principales del juego y aprendimos a usar Godot.

Después empezamos a desarrollar el juego. Implementamos un escenario consistente en un plano y un modelo 3D con forma de cápsula que se movía sobre el plano usando las teclas.

\section*{Semana 3}
Mejoramos el movimiento del personaje: el personaje se movía ahora hasta la posición donde el jugador hiciera click (en vez de con las teclas) y usaba el nodo de Godot de Navigation para ser capaz de llegar a su destino esquivando los obstáculos.

También iniciamos el GDD y establecimos sus bases añadiendo la información que teníamos del juego hasta entonces.

\section*{Semana 4}
Decidimos al primer personaje de nuestro juego, la ingeniera, y decidimos cómo sería su gameplay, habilidades y diseño de personaje.

También creamos el documento de descripción de juego y el elevator pitch, necesarios para la primera entrega de la práctica.

\section*{Semana 5}
Decidimos que, como los integrantes de nuestro grupo no saben hacer animaciones 3D, vamos a cambiar el estilo artístico de nuestro juego de 3D a 2D, usando animaciones y sprites 2D para dar la visión de un "3D falso".

También decidimos que el nombre provisional de nuestro juego sería \emph{SynVic}.

Añadimos el diseño del bardo al GDD.

\section*{Semana 6}
Reunión con el equipo de diseño. Se inicia la creación de un proyectil genérico y se añaden detalles pequeños de programación. Creación de dos habilidades de la ingeniera. Diseño y conexión del HUD del jugador. Se añade el ataque básico al personaje genérico.

\section*{Semana 7}
Terminamos de crear las habilidades de la ingeniera. Les añadimos un tiempo de casteo y solucionamos bugs. Además, cambiamos la forma de procesar los \emph{keyboard inputs} para que se recojan aunque el ratón no se esté moviendo.

Nos reunimos con el equipo de arte y, como están muy ocupados, parece que no podremos contar con su ayuda. Tendremos que encargarnos nosotros mismos del arte del juego.

\section*{Semana 8}	
Añadimos el movimiento de cámara (que puede ser automático o manual), el marcador al modo \emph{DeathMatch}, el menú principal del juego y el modo de juego \emph{Capture}. En este modo de juego, nos faltó añadir el movimiento del hada.

Se actualiza el juego a una versión con networking con el contenido que llevamos hasta el momento.

\section*{Semana 9}
Añadimos el movimiento del hada en el modo \emph{Capture}, completando así este modo de juego.

Se incluye el modo en el networking, y se inicia la creación del Lobby.

\section*{Semana 10}
Añadimos el personaje Mago y creamos el sistema de buffs/debuffs. 

Al igual que en las semanas anteriores, se añaden todas las modificaciones al multijugador.

\section*{Semana 11}
Se completa el Lobby con el sistema de votación y asignación de equipos.
Se inicia una fase de depuración y corrección de errores (bugs).

\section*{Semana 12}
Añadimos modelos 3D y texturas a los obstáculos, personajes y habilidades. También mejoramos el HUD: ahora se muestran visualmente aquellas habilidades que están "activadas" (no se han terminado de castear) y el número de utilizaciones restantes de las habilidades que se pueden castear varias veces antes de ponerse en CD. Por último, se añadieron efectos especiales a la ultimate del mago que muestran el progreso del casteo y el radio de efecto de la habilidad.

Continua la fase de depuración, corrección de bugs y mejora del lobby.
	
\section*{Semana 13}
Añadimos las pantallas de fin de partida, una para el equipo ganador y otra para el perdedor.	

Se sigue con la depuración, se añaden mejoras y se corrigen bugs.
	
\section*{Semana 14}
Depuración y corrección de errores.

\section*{Semana 15}
Añadimos el menú de Descripción de Personajes, que muestra
a los jugadores información sobre su gameplay.

\section*{Semana 16}
Añadimos el audio del juego (música y efectos de sonido) así como el menú de audio.
Depuración y corrección de errores.

\section*{Semana 17}
Depuración y corrección de errores.
Documentación completada (manual y GDD)

\section*{Semana 18}
Minor bugs y pequeñas mejoras, preparación para la entrega.

\end{document}
